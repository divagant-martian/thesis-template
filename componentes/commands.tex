% !TEX root = ../main.tex
% Included by MAIN.TEX

%-------------------------------------------------------------
%                      Own Commands
%-------------------------------------------------------------

% some abreviations ------------------------------------------
\newcommand{\Reg}{$^{\textregistered}$}
\newcommand{\reg}{$^{\textregistered}$ }
\newcommand{\Tm}{\texttrademark}
\newcommand{\tm}{\texttrademark~}
\newcommand {\bsl} {$\backslash$}

% formating --------------------------------------------------
% Theorem & Co environments and counters
\newtheorem{theorem}{Theorem}[chapter]
\newtheorem{lemma}[theorem]{Lema}
\newtheorem{corollary}[theorem]{Corolario}
\newtheorem{remark}[theorem]{Nota}
\newtheorem{definition}[theorem]{Definición}
\newtheorem{equat}[theorem]{Equación}
\newtheorem{example}[theorem]{Ejemplo}
%\newtheorem{algorithm}[theorem]{Algorithm}

% Comentarios sobre el PDF para anotar feedback de las plenarias
\newcommand{\feedbackans}[1]{\marginpar{\pdfcomment[opacity=0.9,hoffset=10pt,open=false,subject={Top2},author={Grupo}]{#1}}}

\renewcommand\descriptionlabel[1]{\normalfont\bfseries{#1}}% Make description environment label bold
% bold and typewritter font tu use as font=\monobold in description environments
\newcommand*{\monobold}[1]{\textbf{\texttt{#1}}}



% Mini lipsum para texto de relleno
\newcommand\tinylipsum{Lorem ipsum dolor sit amet, consectetuer adipiscing elit. Etiam lobortis facilisis sem. Nullam nec mi et neque pharetra sollicitudin}

% Caption near to the thing
\newcommand{\addcaption}[1]{\vspace{-0.2cm}\caption{#1}}

% inserting figures
\newcommand{\insertfigure}[4]{ % Filename, Caption, Label, Width % of textwidth
    \begin{figure}[H]
      \centering
      \includegraphics[width=#4\textwidth]{#1}
      \addcaption{#2}
      \label{#3}
    \end{figure}
}

\definecolor{lighter-gray}{gray}{0.9}
\arrayrulecolor{lighter-gray}
\newcolumntype{L}[1]{>{\raggedright\let\newline\\\arraybackslash\hspace{0pt}}p{#1}}
\newcolumntype{C}[1]{>{\centering\let\newline\\\arraybackslash\hspace{0pt}}p{#1}}
\newcolumntype{R}[1]{>{\raggedleft\let\newline\\\arraybackslash\hspace{0pt}}p{#1}}
\renewcommand\theadfont{\normalsize\bfseries}

\newcommand{\inserttable}[5]{ % Caption, tabular params, the table header, table contents, label
  \vspace{0.2cm}
    \begin{table}[H]
    \begin{center}
      \begin{tabular}{#2}
        \hline
        \rowcolor{lighter-gray}#3\\% encabezados de la tabla
        \toprule
        #4
        \midrule
              \end{tabular}
    \end{center}
    \addcaption{#1}\label{tab:#5}\par
  \end{table}
}

\newcommand{\insertlongtable}[5]{ % Caption, tabular params, the table header, table contents, label
  \vspace{0.2cm}
      \begin{longtable}{#2}
        \hline
        \rowcolor{lighter-gray}#3\\% encabezados de la tabla
        \toprule
        #4
        \midrule
        \caption{#1}\label{tab:#5}
      \end{longtable}
}

\newcommand{\nln}{\newline}

% Referenciar tablas e figuras de forma más estética
\newcommand{\rFig}[1]{Figura \ref{#1}}
\newcommand{\rfig}[1]{figura \ref{#1}}
\newcommand{\rTab}[1]{Tabla \ref{tab:#1}}
\newcommand{\rtab}[1]{tabla \ref{tab:#1}}
\newcommand{\rSec}[1]{Sección \ref{#1}}
\newcommand{\rsec}[1]{sección \ref{#1}}
\newcommand{\rChap}[1]{Capítulo \ref{#1}}
\newcommand{\rchap}[1]{capítulo \ref{#1}}

% comment that appears on the border - very practical !!!
\newcommand{\comment}[1]{\marginpar{\raggedright \noindent \footnotesize {\textsl{#1}} }}

% page clearing
\newcommand{\clearemptydoublepage}{%
  \ifthenelse{\boolean{@twoside}}{\newpage{\pagestyle{empty}\cleardoublepage}}%
             {\clearpage}}

\newcommand{\etAl}{\emph{et al.}\mbox{ }}

\def\fixme{\colorbox{red}{\Large{FIXME}}}
