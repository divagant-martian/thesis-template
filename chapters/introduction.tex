\addchap{Introducción}
\label{chapter:introduction}

La ingeniería de la información ha evolucionado hasta generar técnicas y
tecnologías modernas que generan bondades innegables en las áreas en las que se
implementan. El área de Inteligencia Artificial (AI) está siendo aplicada cada
vez más y con un rango de acción cada vez mayor. Una de las aplicaciones de AI
es el Aprendizaje Automático (Machine Learning o ML), técnica basada en el
aprendizaje computacional automático (supervisado o no supervisado) a partir de
datos de entrada. Algunos de los algoritmos más comunes del Aprendizaje
Automático son: las regresiones lineales, regresiones polinómicas y las redes
neuronales, entre otros.

OVS Group, LLC es una compañía que ofrece una plataforma de automatización de
procesos en la industria petrolera, los servicios son enfocados principalmente
en optimizar la producción de pozos petroleros, hacer más eficiente el uso de
los recursos y en general ayudar a las empresas operadoras a tener un mejor
gerenciamiento de sus campos. Los flujos de trabajo de ingeniería automatizados
en la plataforma OVS se basan en modelos en el “espacio físico” apoyados por
teorías de ingeniería de producción, yacimientos y facilidades. Hasta la fecha
OVS solo ha realizado acercamientos someros a la inclusión de técnicas de AI en
su portafolio comercial.

La posibilidad de añadir las técnicas de Machine Learning, aprovechando la
experiencia técnica y la información a la que tiene acceso OVS, y con el
potencial impacto que puede tener en el ofrecimiento a sus clientes, genera una
clara justificación práctica para realizar un proyecto de viabilidad de la
aplicación de esta tecnología.

Con base en la justificación anterior, el presente trabajo de grado tiene por
objetivo desarrollar una prueba de concepto de la viabilidad de crear un modelo
que use técnicas de Machine Learning para la predicción de tasas petróleo a
partir de parámetros operacionales, con la idea de fortalecer el módulo actual
“Virtual Metering” ofrecido por OVS Group, en específico se crean diferentes
escenarios de pruebas en donde se aplican una variedad de modelos de
aprendizaje supervisado con el fin de validar la factibilidad de éstos en
comparación con las soluciones actuales que se tienen implementadas en la
industria.

Para tener un contexto de los conceptos a los cuales se hace referencia en este
documento, se desarrolla en el segundo capítulo un marco teórico enfocado al
proceso de producción del petróleo.  Teniendo en cuenta la complejidad y la
basta teoría que se ha desarrollado desde hace décadas en esta área, tanto a
nivel de estudios científicos, técnicos como de campo, y dado el enfoque para
el cual se desarrolla este proyecto de grado de maestría, solo se muestra de
forma general y sin profundizar en detalles los fundamentos del sistema
upstream de petróleo, incluyendo la descripción de los sistemas de
levantamiento artificial y las técnicas usadas en las instalaciones de
producción y pruebas. De igual forma, se presentan diferentes referencias del
estado del arte tanto a nivel empresarial con el trabajo que se ha realizado en
diferentes compañías, como a nivel académico y el estudio que se ha realizado
en diferentes papers que han trabajado en temas similares.

El desarrollo del proyecto parte de los datos disponibles y apalancados en las
reglas de negocio de la industria de Oil \& Gas, de las técnicas modernas de
ingeniería de la información y de los puntos críticos identificados y
resueltos, con el objetivo de crear un proceso repetible y semiautomático que
derive datos (predicción/estimación de producción de petrolero), y ofrezca un
reporte o dashboard de control y visualización dentro del framework OVS.
Además, todo el proceso es documentado y abstraído, de manera que se generen
reglas que sean adaptables y replicables con otros datos y otros proyectos de
OVS Group.

Para la implementación de los algoritmos, se usa Keras, que es una API de redes
neuronales de alto nivel, escrita en Python y capaz de ejecutarse sobre
TensorFlow, CNTK o Theano. La ventaja de Keras es que fue desarrollada con un
enfoque que permite la experimentación rápida, ya que, pasar de la idea al
resultado en el menor tiempo posible es clave para realizar una buena
investigación.

Por último, para la gestión del proyecto se usa una metodología ágil de
desarrollo de software como lo es Scrum, enfocando la gestión en las 4 aristas
principales de la solución, que son: la capa de datos, la visualización
(dashboard en OVS) y el entrenamiento y selección del modelo predictivo, y en
la respectiva interconexión entre dichas aristas.
