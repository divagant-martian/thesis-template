\chapter{Contexto}
\label{chapter:context}

\section{Ficha técnica de la organización}
\label{section:ficha} El proyecto se realiza para la empresa OVS Group LLC
(http://ovsgroup.com), compañía de software y servicios dedicada a optimizar
los activos petroleros a través de su framework de software configurable,
llamado OVS.

Fundada en 2009, OVS Group proporciona integración, soluciones automáticas de
flujo de trabajo para clientes de Oil \& Gas (petróleo y gas) que operan en
activos petroleros convencionales y no convencionales en todo el mundo. Como
empresa de software y servicios, se dedican a ofrecer tecnología e innovación
de procesos. Con sede en Houston, Texas, tiene operaciones en Norte América,
Sudamérica, Europa, Medio Oriente y Asia Pacífico. Brinda servicios a clientes
de la industria energética que van desde empresas multinacionales a pequeñas
empresas independientes, que se dedican a la explotación y administración de
activos petroleros, actividad que generar por necesidad natural la integración
y análisis de datos.

El principal producto de la empresa, son los módulos estándares (conocidos en
inglés y dentro de OVS Group como workflows) desarrollados sobre el framework
OVS, los cuales venden y configuran a sus distintos clientes. A continuación,
se describen algunos de ellos como referencia.

Virtual Metering: medir la producción de un solo pozo es un desafío. Los altos
costos y en algunas ocasiones, limitaciones operativas, hacen que la
instalación de medidores de flujo multifásicos en pozos individuales sea poco
práctica. El workflow Virtual Metering llena el vacío en la información,
aprovechando las capacidades de procesamiento e integración de datos de OVS
para conectar automáticamente datos operacionales multifrecuencia con
diferentes modeladores de pozos comerciales. Una fuente virtual ejecuta estos
modelos y devuelve una estimación de producción multifase, que se puede
comparar con producciones prorrateadas y pruebas de pozos esporádicas o
utilizar como entrada para otros procesos de seguimiento de los activos
petroleros.

Well Review Tool: este módulo estándar de OVS Group, permite consolidar e
integrar la información principal de los pozos de una empresa de O\&G,
desplegando toda la información de éstos a través de un dashboard de
ingeniería, con capacidad de profundización (drill-down) para ver detalles en
cada caso.

Production Data Analysis (PDA): es un módulo que proporciona múltiples
diagramas de diagnóstico, gráficos cruzados e histogramas para analizar datos,
para un pozo o grupo de pozos.  Este módulo está diseñado para identificar
valores atípicos en los reservorios que se traducirán en oportunidades para la
optimización de la producción o la identificación de una intervención en el
pozo.

\section{Acta de constitución del proyecto}
\label{section:acta}

\subsection{Definición del problema}
\label{subsection:problema}

\subsection{Definición de objetivos}
\label{subsection:objetivos}

\section{Alcance}
\label{section:alcance}

\section{Situación actual}
\label{section:situacion}

\subsection{Análisis de la situación actual}
\label{subsection:analisis-situacion}

\subsection{Problema que se quiere abordar}
\label{subsection:problema-situacion}

\subsection{Por qué se presenta el problema en la organización}
\label{subsection:justificacion-situacion1}

\subsection{Por qué el problema no se ha resuelto}
\label{subsection:justificacion-situacion2}
