% !TEX root = ../main.tex
% Included by MAIN.TEX

%-------------------------------------------------------------
%                      Own Commands
%-------------------------------------------------------------

% some abreviations ------------------------------------------
\newcommand{\Reg}{$^{\textregistered}$}
\newcommand{\reg}{$^{\textregistered}$ }
\newcommand{\Tm}{\texttrademark}
\newcommand{\tm}{\texttrademark~}
\newcommand {\bsl} {$\backslash$}

% formating --------------------------------------------------
% Theorem & Co environments and counters
\newtheorem{theorem}{Theorem}[chapter]
\newtheorem{lemma}[theorem]{Lema}
\newtheorem{corollary}[theorem]{Corolario}
\newtheorem{remark}[theorem]{Nota}
\newtheorem{definition}[theorem]{Definición}
\newtheorem{equat}[theorem]{Equación}
\newtheorem{example}[theorem]{Ejemplo}
%\newtheorem{algorithm}[theorem]{Algorithm}

% Comentarios sobre el PDF para anotar feedback de las plenarias
\newcommand{\feedback}[1]{\marginpar{\pdfcomment[color=Apricot,opacity=0.9,open=false,subject={Top2},author={Feedback},hoffset=-5pt]{#1}}}
\newcommand{\feedbackans}[1]{\marginpar{\pdfcomment[color=CornflowerBlue,opacity=0.9,hoffset=10pt,open=false,subject={Top2},author={Grupo}]{#1}}}

% Mini lipsum para texto de relleno
\newcommand\tinylipsum{Lorem ipsum dolor sit amet, consectetuer adipiscing elit. Etiam lobortis facilisis sem. Nullam nec mi et neque pharetra sollicitudin}

 \renewcommand\descriptionlabel[1]{\normalfont\bfseries{#1}} % Make description environment label bold

% Caption near to the thing
\newcommand{\addcaption}[1]{\vspace{-0.4cm}\caption{#1}}

% inserting figures
\newcommand{\insertfigure}[4]{ % Filename, Caption, Label, Width % of textwidth
    \begin{figure}[H]
        \begin{center}
            \includegraphics[width=#4\textwidth]{#1}
        \end{center}
        \addcaption{#2}
        \label{#3}
    \end{figure}
}

\newcommand{\inserttable}[4]{ % Caption, tabular params, the table
  \vspace{0.2cm}
  \begin{table}[H]
    \begin{center}
      \begin{tabular}{#2}
        \toprule
        #3
        \bottomrule
      \end{tabular}
    \end{center}
    \addcaption{#1}\label{#4}\par
  \end{table}
}

% Referenciar tablas e figuras de forma más estética
\newcommand{\rFig}[1]{Figura \ref{#1}}
\newcommand{\rfig}[1]{figura \ref{#1}}
\newcommand{\rTab}[1]{Tabla \ref{#1}}
\newcommand{\rtab}[1]{tabla \ref{#1}}
\newcommand{\rSec}[1]{Sección \ref{#1}}
\newcommand{\rsec}[1]{sección \ref{#1}}
\newcommand{\rChap}[1]{Capítulo \ref{#1}}
\newcommand{\rchap}[1]{capítulo \ref{#1}}

% comment that appears on the border - very practical !!!
\newcommand{\comment}[1]{\marginpar{\raggedright \noindent \footnotesize {\textsl{#1}} }}

% page clearing
\newcommand{\clearemptydoublepage}{%
  \ifthenelse{\boolean{@twoside}}{\newpage{\pagestyle{empty}\cleardoublepage}}%
             {\clearpage}}

\newcommand{\etAl}{\emph{et al.}\mbox{ }}
